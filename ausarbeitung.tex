\documentclass[conference]{IEEEtran}
\IEEEoverridecommandlockouts
% The preceding line is only needed to identify funding in the first footnote. If that is unneeded, please comment it out.
\usepackage{cite}
\usepackage{amsmath,amssymb,amsfonts}
\usepackage{algorithmic}
\usepackage{graphicx}
\usepackage{textcomp}
\usepackage{xcolor}
\usepackage{graphicx}
\def\BibTeX{{\rm B\kern-.05em{\sc i\kern-.025em b}\kern-.08em
    T\kern-.1667em\lower.7ex\hbox{E}\kern-.125emX}}
\begin{document}

\title{Untersuchung der Nachweisbarkeit des Akkomodation-Vergenzkonflikts in einem EEG in einer VR entspannungs Szene\\
\thanks{HFU Furtwangen}
}

\author{
	\IEEEauthorblockN{Nick Philipp Häcker}
	\IEEEauthorblockA{\textit{Fakultät Digitale Medien} \\
	\textit{Hochschule Furtwangen}\\
	Furtwangen, Deutschland \\
	haeckern@hs-furtwangen.de}
	
	\and
	\IEEEauthorblockN{Suzan Johannes}
	\IEEEauthorblockA{\textit{Fakultät Digitale Medien} \\
	\textit{Hochschule Furtwangen}\\
	Furtwangen, Deutschland \\
	s.johannes@hs-furtwangen.de}
	\and
	
	\IEEEauthorblockN{Patrick Kaserer}
	\IEEEauthorblockA{\textit{Fakultät Digitale Medien} \\
	\textit{Hochschule Furtwangen}\\
	Furtwangen, Deutschland \\
	patrick.kaserer@hs-furtwangen.de}
	\and
	
	\IEEEauthorblockN{Johann Schulenburg}
	\IEEEauthorblockA{\textit{Fakultät Digitale Medien} \\
	\textit{Hochschule Furtwangen}\\
	Furtwangen, Deutschland \\
	johann.schulenburg@hs-furtwangen.de}
	
	\and
	\IEEEauthorblockN{Lukas Willmann}
	\IEEEauthorblockA{\textit{Fakultät Digitale Medien} \\
	\textit{Hochschule Furtwangen}\\
	Furtwangen, Deutschland \\
	Lukas.willmann@hs-furtwangen.de}
}

\maketitle

\begin{abstract}
This document is a model and instructions for \LaTeX.
This and the IEEEtran.cls file define the components of your paper [title, text, heads, etc.]. *CRITICAL: Do Not Use Symbols, Special Characters, Footnotes, 
or Math in Paper Title or Abstract.
\end{abstract}

\begin{IEEEkeywords}
virtual reality, akkommodation-vergenz konflikt, elektroenzephalografie
\end{IEEEkeywords}

\section{Einleitung}
“Applications of Computer Science, including digital games, virtual reality, and augmented reality [...], have enormous potential for bringing about cultural change.” \cite{b3} Laut einer Prognose von ARtillery Intelligence soll zwischen 2021 bis 2026 der Umsatz von Virtual Reality weltweit von 8,3 auf 28,84 Milliarden US-Dollar steigen. \cite{b2} 
VR Brillen verwenden Stereoskopie, bei welcher jedem Auge leicht unterschiedliche Informationen durch eine horizontale Verschiebung der Bilder zugespielt werden, damit bei dem Nutzer eine 3D Wirkung der Darstellung entsteht. Durch die entstehenden Parallaxen kann die Szene den Eindruck erwecken, aus der Leinwand hervor zu stechen oder dahinter zu liegen.\cite{b4}
Die hier beschriebene Untersuchung konzentriert sich auf den Accommodation Vergenz Konflikt. Dieser tritt gerade in der modernen VR Technologie auf, da durch die Linsen der VR Brille die Bildebene in eine weite Entfernung projiziert wird und die Objekte in VR durch die Interaktion in greifbarer Nähe sind.
“In order to see one object, the eyeballs need to rotate accordingly. This mechanism is called the “vergence”. In a natural situation, as an object is moving closer or further, vergence matches another physiological phenomena: “accommodation”. It enables the object’s image to remain clear on the retina. It is caused by a deformation of the crystalline lens, which focuses light beams the same way camera lenses do.”\cite{b1}
Durch den Versuch soll aufgezeigt werden, ob mögliche Probleme in Anwendungen, welche in dem kritischen Bereich des Akkomodation-Vergenz Konflikt arbeiten, durch das EEG nachweisbar sind oder nicht. Somit kann sich zukünftige Forschung ebenfalls diesem Bereich widmen.

\section{Related Work}
\subsection{Assessing the zone of comfort in stereoscopic displays using EEG}
Im Rahmen des Papers "Assessing the zone of comfort in stereoscopic displays using EEG" \cite{b1} wurde untersucht, wie sich der Akkommodations-Vergenz-Konflikt bei stereoskopischen Anzeigen auf die EEG-Aktivität auswirkt. Das Ziel war es, ein adaptives System zu entwickeln, das mithilfe von leichtgewichtigen EEG-Geräten die Anzeige individuell auf jeden Betrachter kalibrieren kann, um unangenehme Erfahrungen zu vermeiden.\\ 
Eine Pilotstudie wurde durchgeführt, bei der kurze Betrachtungssequenzen verwendet wurden, um Ergebnisse innerhalb eines kurzen Zeitraums vor der Gewöhnung bei den Probanden zu erzielen. Der Versuchsaufbau bestand darin, dass der Proband einen Meter vor einen Bildschirm gesetzt wurde. Dieser zeigte zuerst eine Szene von 2-5 Sekunden in der ein Würfel in Null-Parallaxe auf dem Bildschirm dargestellt wurde. Darauf wurde er zufällig in eine von 9 Positionen 5 Sekunden lang vor oder hinter den Bildschirm stereoskopisch dargestellt. Diese Positionen reichten von 0.212 m vom Probanden entfernt bis 3,046 m vom Probanden entfernt. Sie wurden in Positionen eingeteilt die als komfortabel (C) oder unkomfortabel (NC) klassifiziert wurden. Nach den 5 Sekunden musste der Proband angeben ob der Würfel vor oder hinter dem Bildschirm lag. Dieser Versuch wurde sowohl nur mit Fragebogen als auch mit einer EEG Messung und einem Fragebogen durchgeführt. Bei der EEG Messung gab es 3 Probanden.\\ 
In der Analyse der ereignisbezogenen Potenziale (ERP) ergaben sich Unterschiede zwischen den Bedingungen C und NC. In der NC-Bedingung war die positive Komponente im ERP verzögert, während die negative Komponente in der C-Bedingung eine höhere Amplitude aufwies. Die Untersuchung der Frequenzbänder (ERSP) zeigte ebenfalls Unterschiede zwischen den Bedingungen C und NC. In der NC-Bedingung wurde eine Abnahme der Aktivität im Alpha-Band (10-14 Hz) festgestellt, während eine Zunahme der Aktivität im Theta-Band (4-7 Hz) und Beta-Band (15-25 Hz) im Vergleich zur C-Bedingung beobachtet wurde. Die Probanden schnitten in der unkomfortablen Bedingung besser bei der Tiefenwahrnehmung ab. Die Ergebnisse des Umfragebogens ergaben dabei, dass klares Unwohlsein durch die visuelle Darstellung ausgelöst wurde. Es wurden Widersprüche zu früheren Studien festgestellt, die Stereoskopie mit 2D-Bildern verglichen haben. Es wurden Verbesserungen vorgeschlagen, wie die Messung des Pupillenabstands und die Verwendung fortschrittlicherer Technologien wie VR-Brillen. Die Autoren vermuten, dass die EEG-Messung nicht nur zur Optimierung der Stereoskopie dienen kann, sondern auch zur Entwicklung von Leitlinien für eine verbesserte Technologie.\\
In unserem Versuch können wir unsere EEG-Ergebnisse mit den Ergebnissen dieses Papers bezüglich der EEG-Ergebnisse vergleichen. Unterschiede welche hierbei jedoch hervorzuheben sind dabei, dass die 3D-Szene deutlich komplexer ist als der Aufbau in diesem Paper. Ebenfalls muss man die Anzahl der Versuchspersonen hervorheben und dass in unserem Versuch eine konstante Veränderung der Position des Würfels durchgeführt wird und nicht zu jeder Position der Sehkomfort abgefragt wurde. Bezüglich des technischen Aufbaus muss man berücksichtigen, dass in unserem Versuch beim Einrichten der VR-Brille die IPD der Probanden beachtet wurde und mit der VR-Brille ein deutlich immersiveres Medium verwendet wurde als ein 3D Bildschirm.


\subsection{Study of Electroencephalography-based Objective Stereoscopic Visual Fatigue Evaluation}
In dem Paper "Study of Electroencephalography-based Objective Stereoscopic Visual Fatigue Evaluation" \cite{b5} wird der Bezug zwischen visueller Ermüdung durch den Akkommodations-Vergenz Konflikt bei 3D-Displays hergestellt mittels einer EEG-Messung.\\
Das Experiment wurde an 11 Studenten durchgeführt. Hierbei sollten die Probanden auf einem Random Dot Stereogram feststellen in welche Richtung ein abgebildeter Pfeil zeigt. Es gab 5 minuten Übungszeit und darauf 7mal 10 Minuten die Durchführung des Tests.\\
Die Forscher konnten einen stärkeren Anstieg der Alpha-Wellen im Frontalen Bereich des Gehirns feststellen über die Dauer des Experiments. Die Aktivität der Beta-Wellen stieg auch an, jedoch nicht so stark wie die der Alpha Wellen. Die Forscher stellten in ihrer Forschung fest, dass die Art der Aufgabe auch Einfluss auf die Aktivitäten der Wellen genommen haben kann. Sie schlussfolgern, dass ihr Ergebnis der erhöhten Alpha-Wellen Aktivität gerade im Frontallappen ein Anzeichen für visuelle Ermüdung sein kann.\\
Auch hier können die Ergebnisse des EEGs mit unseren Ergebnissen verglichen werden. Aufgrund der Beschreibung des Versuchsaufbaus wird nicht klar wie stark in dem Versuch der Akkommodation-Vergenz Konflikt erzeugt wurde. Es ist außerdem zu beachten, dass ein Random Dot Stereogram auf einem 3D-Display sehr abweichende Ergebnisse liefern kann im Vergleich zu einer Komplexen 3D Szene auf einem VR-Headset.
 

\section{Methodik}
Lorem ipsum dolor sit amet, consetetur sadipscing elitr, sed diam nonumy eirmod tempor invidunt ut labore et dolore magna aliquyam erat, sed diam voluptua. At vero eos et accusam et justo duo dolores et ea rebum. Stet clita kasd gubergren, no sea takimata sanctus est Lorem ipsum dolor sit amet. Lorem ipsum dolor sit amet, consetetur sadipscing elitr, sed diam nonumy eirmod tempor invidunt ut labore et dolore magna aliquyam erat, sed diam voluptua. At vero eos et accusam et justo duo dolores et ea rebum. Stet clita kasd gubergren, no sea takimata sanctus est Lorem ipsum dolor sit amet.

\section{Ergebnisse}
Hier kommen ein paar Ergebnisse rein...
\begin{figure}[ht]
	\centering
	\includegraphics[width=0.2\textwidth]{assets/alter.png} \hspace{-5pt}
	\includegraphics[width=0.2\textwidth]{assets/ipd.png} \\
	\vspace{2pt}
	\includegraphics[width=0.2\textwidth]{assets/randot.png} \hspace{-5pt}
	\includegraphics[width=0.2\textwidth]{assets/ipd_mvw.png}\\
	\caption{Demografische Daten}
	\label{fig:Demografische Daten}
\end{figure}
Lorem ipsum dolor sit amet, consetetur sadipscing elitr, sed diam nonumy eirmod tempor invidunt ut labore et dolore magna aliquyam erat, sed diam voluptua. At vero eos et accusam et justo duo dolores et ea rebum. Stet clita kasd gubergren, no sea takimata sanctus est Lorem ipsum dolor sit amet. Lorem ipsum dolor sit amet, consetetur sadipscing elitr, sed diam nonumy eirmod tempor invidunt ut labore et dolore magna aliquyam erat, sed diam voluptua. At vero eos et accusam et justo duo dolores et ea rebum. Stet clita kasd gubergren, no sea takimata sanctus est Lorem ipsum dolor sit amet.

\begin{figure}[ht]
	\centering
	\includegraphics[width=0.2\textwidth]{assets/gesch.png} \hspace{-5pt}
	\includegraphics[width=0.2\textwidth]{assets/headset.png} \\
	\vspace{2pt}
	\includegraphics[width=0.2\textwidth]{assets/optBeein.png} 
	\caption{Demografische Daten 2}
	\label{fig:Demografische Daten 2}
\end{figure}
Lorem ipsum dolor sit amet, consetetur sadipscing elitr, sed diam nonumy eirmod tempor invidunt ut labore et dolore magna aliquyam erat, sed diam voluptua. At vero eos et accusam et justo duo dolores et ea rebum. Stet clita kasd gubergren, no sea takimata sanctus est Lorem ipsum dolor sit amet. Lorem ipsum dolor sit amet, consetetur sadipscing elitr, sed diam nonumy eirmod tempor invidunt ut labore et dolore magna aliquyam erat, sed diam voluptua. At vero eos et accusam et justo duo dolores et ea rebum. Stet clita kasd gubergren, no sea takimata sanctus est Lorem ipsum dolor sit amet.

\begin{figure}[ht]
	\centering
	\includegraphics[width=0.2\textwidth]{assets/augenBesch.png} \hspace{-5pt}
	\includegraphics[width=0.2\textwidth]{assets/fokus.png} \\
	\vspace{2pt}
	\includegraphics[width=0.2\textwidth]{assets/schwitz.png} \hspace{-5pt}
	\includegraphics[width=0.2\textwidth]{assets/verschwSicht.png}
	\caption{Fragebogenergebnisse vorher nachher Vergleich}
	\label{fig:Fragebogenergebnisse}
\end{figure}
Lorem ipsum dolor sit amet, consetetur sadipscing elitr, sed diam nonumy eirmod tempor invidunt ut labore et dolore magna aliquyam erat, sed diam voluptua. At vero eos et accusam et justo duo dolores et ea rebum. Stet clita kasd gubergren, no sea takimata sanctus est Lorem ipsum dolor sit amet. Lorem ipsum dolor sit amet, consetetur sadipscing elitr, sed diam nonumy eirmod tempor invidunt ut labore et dolore magna aliquyam erat, sed diam voluptua. At vero eos et accusam et justo duo dolores et ea rebum. Stet clita kasd gubergren, no sea takimata sanctus est Lorem ipsum dolor sit amet.

Hier kommen noch mehr Ergebnisse rein...

\section{Diskussion}
Lorem ipsum dolor sit amet, consetetur sadipscing elitr, sed diam nonumy eirmod tempor invidunt ut labore et dolore magna aliquyam erat, sed diam voluptua. At vero eos et accusam et justo duo dolores et ea rebum. Stet clita kasd gubergren, no sea takimata sanctus est Lorem ipsum dolor sit amet. Lorem ipsum dolor sit amet, consetetur sadipscing elitr, sed diam nonumy eirmod tempor invidunt ut labore et dolore magna aliquyam erat, sed diam voluptua. At vero eos et accusam et justo duo dolores et ea rebum. Stet clita kasd gubergren, no sea takimata sanctus est Lorem ipsum dolor sit amet.

\section{Schlussfolgerung}
Lorem ipsum dolor sit amet, consetetur sadipscing elitr, sed diam nonumy eirmod tempor invidunt ut labore et dolore magna aliquyam erat, sed diam voluptua. At vero eos et accusam et justo duo dolores et ea rebum. Stet clita kasd gubergren, no sea takimata sanctus est Lorem ipsum dolor sit amet. Lorem ipsum dolor sit amet, consetetur sadipscing elitr, sed diam nonumy eirmod tempor invidunt ut labore et dolore magna aliquyam erat, sed diam voluptua. At vero eos et accusam et justo duo dolores et ea rebum. Stet clita kasd gubergren, no sea takimata sanctus est Lorem ipsum dolor sit amet.

\begin{thebibliography}{00}
\bibitem{b1} J. Frey, L. Pommereau, F. Lotte und M. Hachet, 'Assessing the zone of comfort in stereoscopic displays using EEG', ACM SIGCHI Conference on Human Factors in Computing Systems (S. 2041–2046. doi: 10.1145/2559206.2581191. [Online]. Verfügbar unter: https://arxiv.org/pdf/1404.6222)

\bibitem{b2} Statista. Virtual Reality - 'Prognose zum Umsatz weltweit bis 2026' | Statista. https://de.statista.com/statistik/daten/studie/318536/umfrage/prognose-zum-umsatz-mit-virtual-reality-weltweit/ (Zugriff am: 6. Juni 2023).

\bibitem{b3} A. Tatnall, Encyclopedia of education and information technologies. SPRINGER, 2020.

\bibitem{b4} U. Schmidt, Professionelle Videotechnik: Grundlagen, Filmtechnik, Fernsehtechnik (S. 27-28), Geräte- und Studiotechnik in SD, HD, DI, 3D, 6. Aufl. Berlin, Heidelberg: Springer Berlin Heidelberg; Imprint: Springer Vieweg, 2013.

\bibitem{b5} M. Guo, Y. Liu, B. Zou und Y. Wang, 'Study of electroencephalography-based objective stereoscopic visual fatigue evaluation,' in 2015 International Symposium on Bioelectronics and Bioinformatics (ISBB), 2015, S. 160–163, doi: 10.1109/ISBB.2015.7344948.

\end{thebibliography}
\vspace{12pt}
\color{red}
IEEE conference templates contain guidance text for composing and formatting conference papers. Please ensure that all template text is removed from your conference paper prior to submission to the conference. Failure to remove the template text from your paper may result in your paper not being published.

\end{document}
