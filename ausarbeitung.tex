\documentclass[conference]{IEEEtran}
\IEEEoverridecommandlockouts
% The preceding line is only needed to identify funding in the first footnote. If that is unneeded, please comment it out.
\usepackage{cite}
\usepackage{amsmath,amssymb,amsfonts}
\usepackage{algorithmic}
\usepackage{graphicx}
\usepackage{textcomp}
\usepackage{xcolor}
\def\BibTeX{{\rm B\kern-.05em{\sc i\kern-.025em b}\kern-.08em
    T\kern-.1667em\lower.7ex\hbox{E}\kern-.125emX}}
\begin{document}

\title{Untersuchung der Nachweisbarkeit des Akkomodation-Vergenzkonflikts in einem EEG in einer VR entspannungs Szene\\
\thanks{HFU Furtwangen}
}

\author{
	\IEEEauthorblockN{Nick Philipp Häcker}
	\IEEEauthorblockA{\textit{Fakultät Digitale Medien} \\
	\textit{Hochschule Furtwangen}\\
	Furtwangen, Deutschland \\
	nick.philipp.haecker@hs-furtwangen.de}
	
	\and
	\IEEEauthorblockN{Suzan Johannes}
	\IEEEauthorblockA{\textit{Fakultät Digitale Medien} \\
	\textit{Hochschule Furtwangen}\\
	Furtwangen, Deutschland \\
	nick.philipp.haecker@hs-furtwangen.de}
	\and
	
	\IEEEauthorblockN{Patrick Kaserer}
	\IEEEauthorblockA{\textit{Fakultät Digitale Medien} \\
	\textit{Hochschule Furtwangen}\\
	Furtwangen, Deutschland \\
	patrick.kaserer@hs-furtwangen.de}
	\and
	
	\IEEEauthorblockN{Johann Schulenburg}
	\IEEEauthorblockA{\textit{Fakultät Digitale Medien} \\
	\textit{Hochschule Furtwangen}\\
	Furtwangen, Deutschland \\
	johann.schulenburg@hs-furtwangen.de}
	
	\and
	\IEEEauthorblockN{Lukas Willmann}
	\IEEEauthorblockA{\textit{Fakultät Digitale Medien} \\
	\textit{Hochschule Furtwangen}\\
	Furtwangen, Deutschland \\
	Lukas.willmann@hs-furtwangen.de}
}

\maketitle

\begin{abstract}
This document is a model and instructions for \LaTeX.
This and the IEEEtran.cls file define the components of your paper [title, text, heads, etc.]. *CRITICAL: Do Not Use Symbols, Special Characters, Footnotes, 
or Math in Paper Title or Abstract.
\end{abstract}

\begin{IEEEkeywords}
virtual reality, akkommodation-vergenz konflikt, elektroenzephalografie
\end{IEEEkeywords}

\section{Einleitung}
“Applications of Computer Science, including digital games, virtual reality, and augmented reality [...], have enormous potential for bringing about cultural change.” (Tatnall, 2020, 1754) Laut einer Prognose von ARtillery Intelligence soll zwischen 2021 bis 2026 der Umsatz von Virtual Reality weltweit von 8,3 auf 28,84 Milliarden US-Dollar steigen. (ARtillery Intelligence, 2022) 
VR Brillen verwenden Stereoskopie, bei welcher jedem Auge leicht unterschiedliche Informationen durch eine horizontale Verschiebung der Bilder zugespielt werden, damit bei dem Nutzer eine 3D Wirkung der Darstellung entsteht. Durch die entstehenden Parallaxen kann die Szene den Eindruck erwecken, aus der Leinwand hervor zu stechen oder dahinter zu liegen. (Zitat wird benötigt)
Die hier beschriebene Untersuchung konzentriert sich auf den Accommodation Vergenz Konflikt. Dieser tritt gerade in der modernen VR Technologie auf, da durch die Linsen der VR Brille die Bildebene in eine weite Entfernung projiziert wird und die Objekte in VR durch die Interaktion in greifbarer Nähe sind.
“In order to see one object, the eyeballs need to rotate accordingly. This mechanism is called the “vergence”. In a natural situation, as an object is moving closer or further, vergence matches another physiological phenomena: “accommodation”. It enables the object’s image to remain clear on the retina. It is caused by a deformation of the crystalline lens, which focuses light beams the same way camera lenses do.”
Durch den Versuch soll aufgezeigt werden, ob mögliche Probleme in Anwendungen, welche in dem kritischen Bereich des Akkomodation-Vergenz Konflikt arbeiten, durch das EEG nachweisbar sind oder nicht. Somit kann sich zukünftige Forschung ebenfalls diesem Bereich widmen.

\subsection{Related Works}

\section{Methodik}

\section{Ergebnisse}

\section{Diskussion}

\section{Schlussfolgerung}

\section*{References}

Please number citations consecutively within brackets \cite{b1}. The 
sentence punctuation follows the bracket \cite{b2}. Refer simply to the reference 
number, as in \cite{b3}---do not use ``Ref. \cite{b3}'' or ``reference \cite{b3}'' except at 
the beginning of a sentence: ``Reference \cite{b3} was the first $\ldots$''

Number footnotes separately in superscripts. Place the actual footnote at 
the bottom of the column in which it was cited. Do not put footnotes in the 
abstract or reference list. Use letters for table footnotes.

Unless there are six authors or more give all authors' names; do not use 
``et al.''. Papers that have not been published, even if they have been 
submitted for publication, should be cited as ``unpublished'' \cite{b4}. Papers 
that have been accepted for publication should be cited as ``in press'' \cite{b5}. 
Capitalize only the first word in a paper title, except for proper nouns and 
element symbols.

For papers published in translation journals, please give the English 
citation first, followed by the original foreign-language citation \cite{b6}.

\begin{thebibliography}{00}
\bibitem{b1} G. Eason, B. Noble, and I. N. Sneddon, ``On certain integrals of Lipschitz-Hankel type involving products of Bessel functions,'' Phil. Trans. Roy. Soc. London, vol. A247, pp. 529--551, April 1955.
\bibitem{b2} J. Clerk Maxwell, A Treatise on Electricity and Magnetism, 3rd ed., vol. 2. Oxford: Clarendon, 1892, pp.68--73.
\bibitem{b3} I. S. Jacobs and C. P. Bean, ``Fine particles, thin films and exchange anisotropy,'' in Magnetism, vol. III, G. T. Rado and H. Suhl, Eds. New York: Academic, 1963, pp. 271--350.
\bibitem{b4} K. Elissa, ``Title of paper if known,'' unpublished.
\bibitem{b5} R. Nicole, ``Title of paper with only first word capitalized,'' J. Name Stand. Abbrev., in press.
\bibitem{b6} Y. Yorozu, M. Hirano, K. Oka, and Y. Tagawa, ``Electron spectroscopy studies on magneto-optical media and plastic substrate interface,'' IEEE Transl. J. Magn. Japan, vol. 2, pp. 740--741, August 1987 [Digests 9th Annual Conf. Magnetics Japan, p. 301, 1982].
\bibitem{b7} M. Young, The Technical Writer's Handbook. Mill Valley, CA: University Science, 1989.
\end{thebibliography}
\vspace{12pt}
\color{red}
IEEE conference templates contain guidance text for composing and formatting conference papers. Please ensure that all template text is removed from your conference paper prior to submission to the conference. Failure to remove the template text from your paper may result in your paper not being published.

\end{document}
